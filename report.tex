\documentclass[12pt,a4paper]{article}
\usepackage{amsfonts}
\usepackage{amsmath}
\usepackage{amsthm}
\usepackage[utf8]{inputenc}
\usepackage{hyperref}
\usepackage{booktabs}
\usepackage{minted}
\newminted{py}{%
%		linenos,
		fontsize=\small,
		tabsize=2,
		mathescape,
}
\newminted{text}{%
%		linenos,
		fontsize=\small,
		tabsize=2,
		mathescape,
}

\title{Course project}
\author{Shams Methnani \& Rodrigo Arias Mallo}

\begin{document}
\maketitle

\section{Problem statement}


Our problem to solve is to optimally assign nurses to working hours in a 
hospital such that demand for each hour is met and the fewest number of nurses 
are used.

This is an NP-hard problem in combinatorial optimization. It can be formulated 
as an integer program.

We are given a set of nurses, a specification of the demands for each hour as 
well as a set of constraints that must be satisfied regarding how long each 
nurse can work.

\subsection{Parameters}

We are given a set of available nurses of size $N$, a list ${demand}_h$ of size 
$H$ with the number of nurses needed each hour, as well as the constraints that 
must be satisfied. The parameters are given as follows:
%
\begin{center}
\begin{tabular}{rl}
$demand_h$ 			&	Demand for each hour $h$ \\
$maxHours$ 			& Max. number of hours each nurse can work \\
$minHours$ 			& Min. number of hours each nurse can work \\
$maxConsec$ 		& Max. consecutive hours each nurse can work \\
$maxPresence$ 	& Max. hours present at the hospital\\
\end{tabular}
\end{center}
%
\subsection{Decision variables}

For the problem, we define three matrices as decision variables. The last one is
auxiliary.
%
\begin{table}[h]
\centering
\begin{tabular}{c l}
\toprule
Symbol & Type and size \\
\midrule
$P$ & Binary matrix of size $N \times H$.\\
$W$ & Binary matrix of size $N \times H$.\\
$T$ & Binary matrix of size $N \times H$.\\
\bottomrule
\end{tabular}
\end{table}

\noindent
The matrix $P$ has the element $P_{n,h} = 1$ if the nurse $n$ is present at the hospital
at hour $n$, also if is working, $W_{n,h} = 1$, otherwise 0.
%
The matrix $T$ is an auxiliary matrix, with the element $T_{n,h} = 1$ if the
nurse $n$ is travelling to the hospital at the hour $h$, otherwise 0.

\subsection{Constraints}

\paragraph{Constraint 1} At least $\textrm{demand}_h$ nurses should be working
at the hour $h$.
$$ \forall h\in [1, H],\, \sum_{n \in N} W_{n,h} \ge \textrm{demand}_h$$
%
\paragraph{Constraint 2} Each working nurse should work at least minHours.
$$ \forall n \in N,\, \sum_{h \in [1, H]} W_{n,h} \ge
  \textrm{minHours} * \sum_{j \in [1, H]} T_{n,j} $$
%
\paragraph{Constraint 3} Each nurse should work at most maxHours.
$$ \forall n \in N,\, \sum_{h \in [1, H]} W_{n,h} \le \textrm{maxHours} $$
%
\paragraph{Constraint 4a} Each nurse should work at most maxConsec consecutive
hours.
$$ \forall n \in N,\, \forall h \in [1, \textrm{H} - \textrm{maxConsec}],$$
$$\sum_{k \in [0, maxConsec]} W_{n, h+k} \le \textrm{maxConsec}
$$
%
\paragraph{Constraint 4b} Wrap around case for maxConsecutive hours.
$$ \forall n \in N,\, \forall h \in [1, \textrm{maxConsec}],$$
$$\sum_{j \in [1, h]} W_{n, j} +
  \sum_{j \in [H-maxConsec+h, H]} W_{n, j}  \le \textrm{maxConsec}
$$
%
\paragraph{Constraint 5} No nurse can stay at the hospital for more than
maxPresence hours.
$$ \forall n \in N, \, \sum_{h \in [1, H]} P_{n,h} \le \textrm{maxPresence} $$
%
\paragraph{Constraint 6} No nurse can rest for more than one consecutive hour.
$$ \forall n \in N,\, \forall h \in [1, \textrm{H-1}],$$
$$W_{n, h} + W_{n, h+1} \ge P_{n,h+1} $$
%
\paragraph{Constraint 7} Working nurses can travel to hospital at most once.
$$ \forall n \in N,\,$$
$$ \sum_{h \in [1, H]} T_{n,h} \le 1 $$
%
\paragraph{Constraint 8a} If a nurse is present at current hour and wasn't at
previous hour, then they traveled to hospital at current hour.
$$ \forall n N, \,\forall h \in [2, \textrm{H}],\,$$
$$ T_{n,h} \ge 1 - P_{n, h-1} + P_{n, h} - 1 $$
%
\paragraph{Constraint 8b} Add case for wrap-around
$$ \forall n \in N,\,$$
$$ T_{n, 1} \ge 1 - P_{n, H} + P_{n, 1} - 1 $$
%
\paragraph{Constraint 9} If nurse travels at current hour, then they were not
present at previous hour.
$$ \forall n \in N,\,\forall h \in [2, \textrm{H}],\,$$
$$ T_{n,h} \le 1 - P_{n, h-1} $$
%
\paragraph{Constraint 10} If nurse travels at current hour, then they are
present at current hour.
$$ \forall n \in N,\,\forall h \in [1, H],\,$$
$$ T_{n,h} \le P_{n, h} $$
%
%
\section{GRASP}
%
The GRASP method is a metaheuristic in which each iteration consists of two
phases: construction and local search.

First, we need to specify the problem in terms of parts that can be selected.  
The ground set $E$ contains elements that can be selected in order to build a 
solution. Let $S$ be a possible solution by selecting parts of $E$, so $S \in 
2^E$. We can build $F$ as the set of solutions that are feasible $F \subseteq 
2^E$

In our problem, we need to select the behavior of $N$ nurses. A possible 
solution can be defined by the selection of elements that describe what a nurse 
$n$ is doing at the time $h$. A tuple $t = (n, h, s)$ can code that the nurse 
$n$ should be at state $s$ in the time $h$. The state can be 0 if the nurse is 
not at the hospital, 1 if is working, or 2 if is resting. A feasible solution 
$S$ should consist of a selection of tuples $t$ such that, they define the 
behavior of every nurse at every time, and also satisfy the previous 
constraints.

Furthermore, in order to build a solution, a notion of cost is needed to select 
among all the possible tuples that can be selected. Lets ignore the constraints 
in the cost function, and check the feasibility at the end. Iteratively we can 
select a naive solution by a simple procedure.


We begin with an empty solution, $S=\{\}$. Then we compute the cost of all the 
possible elements that can be added to the solution. A good candidate will be 
try to meet the demand by adding nurses that are in the working state.

We create the restricted candidate list (RCL) by selecting working nurses with 
high probability. Then, one of the possible combinations will be chosen, 
optimistically assigning one nurse to work in that time.  Then we continue 
adding nurses in a working state.  Once the demand is met, then we now will 
prefer adding to the solution tuples that contain nurses in a non-working state.  
The probability of such state should now be added with high probability.

The selection of the cost function is the core of the GRASP algorithm, so we
need to take some care to guarantee a good behavior.
%
\section{BRKGA}
%
Genetic algorithms borrow mechanisms from nature, namely the concept of survival
of the fittest, in order to search a large solution space. A population of
solutions is evolved over several generations by iteratively breeding new
solutions by mating or \textit{crossing} individuals from previous generations.
Biased Randomized-Key Genetic Algorithm (BRKGA) is such an algorithm in which
certain members of the population are selected for crossover with higher
probability.

We represent our problem with a set of individuals with an associated
\textit{chromosome}, each of which encodes a solution.

Each individual or chromosome has associated with it a fittness score, which
will determine if they are an \textit{elite individual} or \textit{non-elite
individual}. We can now add a bias to the selection of the parentage for next
generation individuals the selecting a parent from the elite set with a higher
probability in order to pass on desireable characteristics.

Each chromosome is represented as a string of genes. Each gene takes on a
particular value from an alphabet.

Our matrix encodes start-hour, present-hour, number-breaks, break-score
3 scalars and an array of scores in order to place breaks.

% Each gene is a vector of size N\times(3+H), the first row

\subsection{Chromosome structure}

The solution is codified by using three scalars $p$, $s$ and $b$ and a vector
$v$ of size $H$ for each nurse. The scalars codify the number of hours $p$ that
the nurse is present at the hospital, the first hour $s$ at which is at the
hospital, and the number of breaks $b$ while is present.

The vector $v$ is used from $1$ to the number of present hours $p$. Each element
stores a score that will be used to determine the preferred hours in which a
break should be set. Lower scores will be selected first, filling the breaks up
to $b$.

The decision whether a nurse works or not, is based on $p$, if the presence
hours are less than $minHours$ the nurse doens't start working.

The choromosome structure is then matrix of $N$ rows and $3+H$ columns, but
flattened into a long vector of size $N \times 3+H$.

\subsection{Decoder}

The decoder part of the algorithm, which is the only part that depends on the 
problem, computes the fitness of each individual in the population. The fitness 
$f(p,s,b,v)$ is a heuristic value that estimates the objective, and has to be 
minimized. Is computed by adding five badness values $BN$, $BD$, $BH$, $BC$ and 
$BF$ together.

In order to compute the values, we first need to determine when each nurse 
works. By using $p$ and $s$ we have the interval that they are present.  And 
using the indices of the first $b$ lower values of $v$, we now where they rest.

The number of nurses needed is the first value $BN$, which takes into account 
that we want to minimize the number of nurses used.

The value $BD$ measures the demand left that is not met. Is computed at each 
hour $h$ as the difference of the problem demand and the nurses offer (the 
working hours provided by the working nurses) as $BD = 100 \sum_h 
\textrm{demand}_h - \textrm{offer}_h$.

The value $BH(n)$ measures the error of each nurse $n$, when they work more than 
$maxHours$ or when they work less than $minHours$, we use $w = p - b$ to compute 
the number of working hours. If they work more hours than allowed, $BH(n)$ is 
set to the number of exceeding hours $w - maxHours$. If they work less than 
allowed, $BH(n)$ is set to the number of hours left $minHours - w$.  Finally, we 
get $BH = \sum_n 200 + 20 BH(n)$.

Note that we use the floating point representation of the values $p$, $s$, $b$ 
and $v$ when they are part of the fitness function. So we can allow that lower 
differences of each gene produce a change in the fitness function. Otherwise, we 
use the integer representation of the parameters.

The value $BC$ test if there is any problem with the maximum number of 
consecutive hours that each nurse is working. For each nurse $n$ the maximum 
number of consecutive hours $M$ is computed, and if is greater than $maxConsec$, 
$BC(n)$ is set to $100(M - maxConsec) - 10b$. We use the breaks to prefer those 
solutions that include more breaks. Finally we have $BC = \sum_n BC(n)$.

The value $BF$ measures the distance of the non-working nurses to the hours that 
have some demand left. The distance is computed for each hour with demand left 
$h$ to the closest hour $c$ that the nurse would start working if $p$ was at 
least $minHours$. Is the start hour $s$ when $s > h$ and the end hour $s+p$ when 
$s+p < h$. So, if the nurse don't cover the hour $h$, $BF(n) = h - c$. However, 
if the nurse could cover the hour, without changing start, we only need to take 
into account the distance to minHours, so $BF(n) = minHours - p$. The final 
value is the sum for all nurses that are non-working, $BF = \sum_n BF(n)$

The final fitness value is just the sum of all the intermediate values:

$$ f(p,s,b,v) = BN + BD + BH + BC + BF $$

The fitness is stored in an array, computed for each individual of the 
population, and returned to the BRKGA algorithm, in order to continue the 
iteration.

\section{Results}

The three solvers were compared solving equal problems, with a big number of 
nurses.

\begin{table}[h]
%\footnotesize
\centering
\begin{tabular}{cccccccc}
\toprule
       &          &\multicolumn{2}{c}{ILP} & \multicolumn{2}{c}{GRASP} & \multicolumn{2}{c}{BRKGA} \\
Seed   &  Nurses  & Obj. & Time & Obj. & Time & Obj. & Time \\
\midrule
     8 &     90   &     41   & 61.1    &       52  &   62.9      &     ?      &  83.1809 \\
     9 &     90   &     39   & 61.2    &       ?   &  619.3      &     42     &  71.2341 \\
    10 &     90   &     39   & 61.2    &       56  &   62.2      &     42     &  51.6024 \\
    11 &     90   &     31   & 61.2    &       43  &   63.4      &     44     &  46.2278 \\
    12 &     90   &     39   & 61.2    &       46  &  127.4      &     41     &  96.5402 \\
    13 &     90   &     36   & 61.1    &       ?   &  638.2      &     ?      &  81.3786 \\
    14 &     90   &     39   & 61.1    &       51  &   95.3      &     ?      &  71.1666 \\
    15 &     90   &     31   & 61.2    &       42  &   32.2      &     43     &  65.3363 \\
    16 &     90   &     35   & 61.1    &       56  &  189.3      &     44     &  45.5936 \\
    17 &     90   &     35   & 61.2    &       57  &   31.5      &     42     &  63.4561 \\
\bottomrule
\end{tabular}
\caption{Results with the different methods}
\label{tab:res}
\end{table}

\section{Structure of the project}

\subsection{The OPL model}

The ILP problem was first written in OLP by using a small example, and it can be 
found in the \texttt{opl/} folder. The \texttt{proj.mod} file contains the model 
of the problem, and the example data is stored in \texttt{proj.dat}. A small 
python script and a CPLEX batch file help the extraction of the solution. The 
Makefile is designed to use \texttt{oplrun} and \texttt{cplex} to solve the 
problem and present the solution. The \texttt{oplrun} program evaluates the 
model and builds the file \texttt{sol.lp}, which is solved by the \texttt{cplex} 
program. The script \texttt{cplex-save.txt} provide CPLEX the instructions to 
save the solution in \texttt{sol.xml} in order to be read by the python script 
\texttt{cplex-save.txt}:

\begin{textcode}
% cd opl/
% make
...
cplex -f cplex-save.txt

Welcome to IBM(R) ILOG(R) CPLEX(R) Interactive Optimizer 12.6.0.0
...
Root relaxation solution time = 0.00 sec. (0.15 ticks)
...
python read-sol.py
NStart
[[0 0 0 0 1 0]
 [1 0 0 0 0 0]
 [1 0 0 0 0 0]]
NWorking
[[0 0 0 0 1 1]
 [1 1 0 0 0 0]
 [1 1 0 0 0 0]]
NPresent
[[0 0 0 0 1 1]
 [1 1 0 0 0 0]
 [1 1 0 0 0 0]]
\end{textcode}

As we need to run and evaluate a lot of problems, this method is a bit complex, 
as requires the problem to be defined in the OPL data language.  We opted for 
the use of the PuLP python package to communicate with CPLEX, in order to write 
the problem and read the solution as well as the objective value, more easily.

\subsection{The PuLP model}

The PuLP package provides a almost transparent interface to work with multiple 
solvers. We used the CPLEX solver by default, as it seems to be the fastest.

The same constraints in \texttt{opl/proj.mod} were translated into python. The 
file \texttt{src/lp.py} contains the ILP model in python. A class 
\texttt{Solution} provides a generic method \texttt{solve} to be implemented, as 
well as an instance to a problem to be solved.

The instance generator is built in the \texttt{Problem} class, located in 
\texttt{nurses.py}

\subsection{Instance generator}

Random instances of the problem are generated by using the parameters: 
\texttt{seed} the random seed, \texttt{N} the number of nurses and \texttt{H} 
the hours in a day, set by default to 24.

We used a random distribution, with more expected value in the daily hours, 
behaving similarly to a real case. Once a instance problem generated can be 
solved by any solver. An example problem:

\begin{textcode}
% python nurses.py
Problem(seed=1, N=10, H=24)
-------------------------------------------------------
hour  0 0 0 0 0 0 0 0 0 0 1 1 1 1 1 1 1 1 1 1 2 2 2 2 
      0 1 2 3 4 5 6 7 8 9 0 1 2 3 4 5 6 7 8 9 0 1 2 3 
-------------------------------------------------------
dem.  1 1 0 0 1 0 0 2 1 2 2 2 3 2 3 1 1 0 1 0 0 0 1 0 
-------------------------------------------------------
 maxConsec   : 7
 maxPresence : 9
 maxHours    : 8
 minHours    : 8
-------------------------------------------------------
\end{textcode}

%\subsection{Solvers}


\section{How to run}
The ILP, GRASP and BRKGA solvers are coded in a extension of the 
\texttt{Solution} class, placed in \texttt{src/}, in the files \texttt{lp.py}, 
\texttt{grasp.py} and \texttt{brkga.py}, respectively.

Each solver is iteratively choosen to solve the same problem. Once all the 
solutions are computed, a new random problem is generated and solved again. The 
file \texttt{main.py} produces the table~\ref{tab:res}. Can be executed as:
%
\begin{textcode}
% python main.py
\end{textcode}

\end{document}
