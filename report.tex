\documentclass[12pt,a4paper]{article}
\usepackage{amsfonts}
\usepackage{amsmath}
\usepackage{amsthm}
\usepackage[utf8]{inputenc}
\usepackage{hyperref}
\usepackage{booktabs}

\title{Course project}
\author{Shams Methnani \& Rodrigo Arias Mallo}

\begin{document}
\maketitle

\section{Problem statement}


Our problem to solve is to optimally assign nurses to working hours in a hospital such that demand for each hour is met and the fewest number of nurses are used.

This is an NP-hard problem in combinatorial optimization. It can be formulated as an integer program.

We are given a set of nurses, a specification of the demands for each hour as well as a set of constraints that must be satisfied regarding how long each nurse can work.

\subsection{Parameters}

We are given a set of available nurses of size $N$, a list ${demand}_h$ of size $H$ with the number of nurses needed each hour, as well as the constraints that must be satisfied.

The parameters are given as follows:
    \begin{enumerate}
    \paragraph{}
        \item [${demand}_h$] a list of demands for each hour
        \item [${maxHours}$] maximum number of hours each nurse may work
        \item [${minHours}$] minimum number of hours each nurse may work
        \item [${maxConsec}$] maximum number of consecutive hours each nurse may work
        \item [${maxPresence}$] maximum number of hours each nurse may be present at the hospital after arriving
    \end{enumerate}
%
\subsection{Decision variables}

For the problem, we define three matrices as decision variables. The last one is
auxiliary.
%
\begin{table}[h]
\centering
\begin{tabular}{c l}
\toprule
Symbol & Type and size \\
\midrule
$P$ & Binary matrix of size $N \times H$.\\
$W$ & Binary matrix of size $N \times H$.\\
$T$ & Binary matrix of size $N \times H$.\\
\bottomrule
\end{tabular}
\end{table}

\noindent
The matrix $P$ has the element $P_{n,h} = 1$ if the nurse $n$ is present at the hospital
at hour $n$, also if is working, $W_{n,h} = 1$, otherwise 0.
%
The matrix $T$ is an auxiliary matrix, with the element $T_{n,h} = 1$ if the
nurse $n$ is travelling to the hospital at the hour $h$, otherwise 0.

\subsection{Constraints}

\paragraph{Constraint 1} At least $\textrm{demand}_h$ nurses should be working
at the hour $h$.
$$ \forall h\in [1, H],\, \sum_{n \in N} W_{n,h} \ge \textrm{demand}_h$$
%
\paragraph{Constraint 2} Each working nurse should work at least minHours.
$$ \forall n \in N,\, \sum_{h \in [1, H]} W_{n,h} \ge
  \textrm{minHours} * \sum_{j \in [1, H]} T_{n,j} $$
%
\paragraph{Constraint 3} Each nurse should work at most maxHours.
$$ \forall n \in N,\, \sum_{h \in [1, H]} W_{n,h} \le \textrm{maxHours} $$
%
\paragraph{Constraint 4a} Each nurse should work at most maxConsec consecutive
hours.
$$ \forall n \in N,\, \forall h \in [1, \textrm{H} - \textrm{maxConsec}],$$
$$\sum_{k \in [0, maxConsec]} W_{n, h+k} \le \textrm{maxConsec}
$$
%
\paragraph{Constraint 4b} Wrap around case for maxConsecutive hours.
$$ \forall n \in N,\, \forall h \in [1, \textrm{maxConsec}],$$
$$\sum_{j \in [1, h]} W_{n, j} +
  \sum_{j \in [H-maxConsec+h, H]} W_{n, j}  \le \textrm{maxConsec}
$$
%
\paragraph{Constraint 5} No nurse can stay at the hospital for more than
maxPresence hours.
$$ \forall n \in N, \, \sum_{h \in [1, H]} P_{n,h} \le \textrm{maxPresence} $$
%
\paragraph{Constraint 6} No nurse can rest for more than one consecutive hour.
$$ \forall n \in N,\, \forall h \in [1, \textrm{H-1}],$$
$$W_{n, h} + W_{n, h+1} \ge P_{n,h+1} $$
%
\paragraph{Constraint 7} Working nurses can travel to hospital at most once.
$$ \forall n \in N,\,$$
$$ \sum_{h \in [1, H]} T_{n,h} \le 1 $$
%
\paragraph{Constraint 8a} If a nurse is present at current hour and wasn't at
previous hour, then they traveled to hospital at current hour.
$$ \forall n N, \,\forall h \in [2, \textrm{H}],\,$$
$$ T_{n,h} \ge 1 - P_{n, h-1} + P_{n, h} - 1 $$
%
\paragraph{Constraint 8b} Add case for wrap-around
$$ \forall n \in N,\,$$
$$ T_{n, 1} \ge 1 - P_{n, H} + P_{n, 1} - 1 $$
%
\paragraph{Constraint 9} If nurse travels at current hour, then they were not
present at previous hour.
$$ \forall n \in N,\,\forall h \in [2, \textrm{H}],\,$$
$$ T_{n,h} \le 1 - P_{n, h-1} $$
%
\paragraph{Constraint 10} If nurse travels at current hour, then they are
present at current hour.
$$ \forall n \in N,\,\forall h \in [1, H],\,$$
$$ T_{n,h} \le P_{n, h} $$
%
%
\section{GRASP}
%
The GRASP method is a metaheuristic in which each iteration consists of two
phases: construction and local search.

First, we need to specify the problem in terms of part that can be selected. The
ground set $E$ contains elements that can be selected in order to build a
solution. Let $S$ be a possible solution by selecting parts of $E$, so $S \in
2^E$. We can build $F$ as the set of solutions that are feasible $F \subseteq
2^E$

In our problem, we need to select the behavior of $N$ nurses. A possible solution
can be defined by the selection of elements that describe what a nurse $n$ is
doing at the time $h$. A tuple $t = (n, h, s)$ can code that the nurse $n$
should be at state $s$ in the time $h$. A feasible solution $S$ should consist
of a selection of tuples $t$ such that, they define the behavior of every nurse
at every time, and also satisfy the previous constraints.

Furthermore, in order to build a solution, a notion of cost is needed to select
among all the possible tuples that can be selected. Lets ignore the constraints
in the cost function, and check the feasibility at the end. Iteratively we can
select a naive solution by a simple procedure.


We begin with an empty solution, at time $h = 1$. At that time, we have a demand $d$ given by
$\textrm{demand}_h$. A good candidate will be try to meet the demand by adding nurses that are already working. We create the restricted candidate list (RCL) by selecting working nurses with high probability. Then, one of the possible
combinations will be chosen, optimistically assigning one nurse to work in that time.
Then we continue adding nurses in a working state. Once the demand is met, then
we now will prefer adding to the solution tuples that contain nurses in a
non-working status. The probability of such state should now be added with high
probability.

The selection of the cost function is the core of the GRASP algorithm, so we
need to take some care to guarantee a good behavior.

\section{BRKGA}
%
Genetic algorithms borrow mechanisms from nature, namely the concept of survival of the fittest, in order to search a large solution space. A population of solutions is evolved over several generations by iteratively breeding new solutions by mating or \textit{crossing} individuals from previous generations. Biased Randomized-Key Genetic Algorithm (BRKGA) is such an algorithm in which certain members of the population are selected for crossover with higher probability.

We represent our problem with a set of individuals with an associated \textit{chromosome}, each of which encodes
a solution.

Each individual or chromosome has associated with it a fitness score, which will determine if they are an \textit{elite individual} or \textit{non-elite individual}. We can now add a bias to the selection of the parentage for next generation individuals by the selecting a parent from the elite set with a higher probability in order to pass on desireable characteristics.

Each chromosome is represented as a string of genes. Each gene takes on a particular value
from an alphabet.

Our matrix encodes start-hour, present-hour, number-breaks, break-score
3 scalars and an array of scores in order to place breaks.

% Each gene is a vector of size N\times(3+H), the first row

\end{document}
