\documentclass[12pt,a4paper]{article}
\usepackage{amsfonts}
\usepackage{amsmath}
\usepackage{amsthm}
\usepackage[utf8]{inputenc}
\usepackage{hyperref}
\usepackage{booktabs}
\usepackage{float}
\usepackage{minted}
\newminted{py}{%
%		linenos,
		fontsize=\small,
		tabsize=2,
		mathescape,
}
\newminted{text}{%
%		linenos,
		fontsize=\small,
		tabsize=2,
		mathescape,
}

\title{Course project}
\author{Shams Methnani \& Rodrigo Arias Mallo}

\begin{document}
\maketitle

\section{Problem statement}


Our problem to solve is to optimally assign nurses to working hours in a 
hospital such that demand for each hour is met and the fewest number of nurses 
are used.

This is an NP-hard problem in combinatorial optimization. It can be formulated 
as an integer program.

We are given a set of nurses, a specification of the demands for each hour as 
well as a set of constraints that must be satisfied regarding how long each 
nurse can work.

\subsection{Parameters}

The problem is described by the following set of parameters:
%
\begin{center}
\begin{tabular}{rl}
$N$             & The number of nurses available \\
$H$             & The hours in a day \\
$demand_h$ 			&	Demand for each hour $h$ \\
$maxHours$ 			& Max. number of hours each nurse can work \\
$minHours$ 			& Min. number of hours each nurse can work \\
$maxConsec$ 		& Max. consecutive hours each nurse can work \\
$maxPresence$ 	& Max. hours present at the hospital\\
\end{tabular}
\end{center}
%
\subsection{Decision variables}

For the problem, we used three binary matrices $P$, $W$ and $S$ of size $N 
\times H$ as decision variables.  The last one is auxiliary. The matrix $P$ has 
the element $P_{n,h} = 1$ if the nurse $n$ is present at the hospital
at hour $n$, also if is working, $W_{n,h} = 1$, otherwise 0. The matrix $S$ is 
an auxiliary matrix, with the element $S_{n,h} = 1$ if the nurse $n$ starts at 
the hospital at the hour $h$, the first hour that is present, otherwise 0.

\subsection{Formal definition}

Minimize the objective function:
%
$$\sum_{n=1}^N \sum_{h=1}^H S_{n,h} $$
%
Subject to the following constraints:
%
\paragraph{Constraint 1} At least $demand_h$ nurses should be working
at the hour $h$.
$$ 1 \le h \le H,\quad \sum_{n=1}^N W_{n,h} \ge demand_h$$
%
\paragraph{Constraint 2} Each working nurse should work at least $minHours$.
$$ 1 \le n \le N,\quad \sum_{h = 1}^H W_{n,h} \ge
	minHours \cdot \sum_{j=1}^H S_{n,j} $$
%
\paragraph{Constraint 3} Each nurse should work at most $maxHours$.
$$ 1 \le n \le N,\quad \sum_{h = 1}^H W_{n,h} \le maxHours $$
%
\paragraph{Constraint 4} Each nurse should work at most $maxConsec$ consecutive
hours.
$$ 1 \le n \le N,\quad 1 \le h \le H - maxConsec, \quad 1 \le k \le maxConsec,$$
$$\sum_k W_{n, h+k} \le maxConsec
$$
%
\paragraph{Constraint 5} No nurse can stay at the hospital for more than
$maxPresence$ hours.
$$ 1 \le n \le N, \quad \sum_{h = 1}^H P_{n,h} \le maxPresence $$
%
\paragraph{Constraint 6} No nurse can rest for more than one consecutive hour.
$$ 1 \le n \le N,\quad  1 \le h \le H-1,$$
$$ P_{n,h} - W_{n, h} + P_{n,h+1} - W_{n, h+1} \le 1 $$
%
\paragraph{Constraint 7} Working nurses can start at most once.
$$ 1 \le n \le N,\quad \sum_{h = 1}^H S_{n,h} \le 1 $$
%
\paragraph{Constraint 8a} If a nurse is present at hour $h$ and wasn't at
previous hour, then started at current hour $h$.
$$ 1 \le n \le N, \quad 2 \le h \le H,$$
$$ S_{n,h} \ge 1 - P_{n, h-1} + P_{n, h} - 1 $$
%
\paragraph{Constraint 8b} Add case for the first hour, if is present, has 
started at the first hour.
$$ 1 \le n \le N,$$
$$ S_{n, 1} = P_{n, 1} $$
%
\paragraph{Constraint 9} If a nurse starts at current hour, then was not
present at previous hour.
$$ 1 \le  n \le N ,\quad 2 \le h \le H,$$
$$ S_{n,h} \le 1 - P_{n, h-1} $$
%
\paragraph{Constraint 10} If a nurse starts at current hour, then should be
present at current hour.
$$ 1 \le n \le N,\quad 2 \le h \le H,$$
$$ S_{n,h} \le P_{n, h} $$
%
%
\section{GRASP}
%
The GRASP method is a meta-heuristic in which each iteration consists of two 
phases: construction and local search.

First, we need to specify the problem in terms of parts that can be selected.  
The ground set $E$ contains elements that can be selected in order to build a 
solution. Let $S$ be a possible solution by selecting parts of $E$, so $S \in 
2^E$. We can build $F$ as the set of solutions that are feasible $F \subseteq 
2^E$

In our problem, we need to select the behavior of $N$ nurses. A possible 
solution can be defined by the selection of elements that describe what a nurse 
$n$ is doing at the time $h$. A tuple $t = (n, h, s)$ can code that the nurse 
$n$ should be at state $s$ in the time $h$. The state can be 0 if the nurse is 
not at the hospital, 1 if is working, or 2 if is resting. A feasible solution 
$S$ should consist of a selection of tuples $t$ such that, they define the 
behavior of every nurse at every time, and also satisfy the previous 
constraints.

Furthermore, in order to build a solution, a notion of cost is needed to select 
among all the possible tuples that can be selected. Lets ignore the constraints 
in the cost function, and check the feasibility at the end. Iteratively we can 
select a naive solution by a simple procedure.


%
\subsection{Constructive phase}
%
We begin with an empty solution. Then we compute the cost of all the possible 
elements that can be added to the solution. A good candidate will try to meet 
the demand by adding nurses that are in the working state.

We create the restricted candidate list (RCL) by selecting working nurses with 
high probability. Then, one of the possible combinations will be chosen, 
optimistically assigning one nurse to work in that time.  Then we continue 
adding nurses in a working state.  Once the demand is met, then we now will 
prefer adding to the solution tuples that contain nurses in a non-working state.  
The probability of such state should now be added with high probability.

A small snippet of a simplified python pseudocode can be found below, describing 
the constructive phase as the \texttt{greedy\_build} function.
%
\begin{pycode}
def greedy_build(self, alpha):
	solution = set()
	C = self.initial_candidates()
	while len(C) != 0:
		costs = self.evaluate_costs(C, solution)
		cmin = np.min(costs)
		cmax = np.max(costs)
		cmed = cmin + alpha * (cmax - cmin)
		RCL_indices = np.where(costs <= cmed)[0]
		si = np.random.choice(RCL_indices)
		s = C[si]

		self.select_candidate(solution, s)
		C = self.update_candidates(C, s)
	return solution
\end{pycode}
%
We see the algorithm follows closely the definition of the GRASP meta-heuristic.  
The RCL equation, is almost the same.
$$
	RCL = \{e \in C \mid c(e) \le c_{min} + \alpha (c_{max} - c_{min})\}
$$
The selection of the cost function is the core of the GRASP algorithm, so we
need to take some care to guarantee a good behavior.

\subsection{Cost function}

The function was designed to select at each step the most promising element. Is 
based on a set of rules to assign large weights to infeasible actions. All the 
elements in the set of candidates $C$ are evaluated. A simplified version is 
shown below.
%
\begin{pycode}
	def evaluate_costs(self, C, solution):
		...
		costs = np.zeros([len(C)], dtype='int')
		infos = []
		inf = 1000

		for i in range(len(C)):
			e = C[i]
			cost = 0
			info = ''
			n, h, s = e

			already_assigned = (n,h) in self.assigned

			working_time = self.worktime[n]

			present_hours = self.presencetime[n]

			work_done = self.workdone[h]
			demand_left = p.demand[h] - work_done
			hours_left = p.minHours - work_done

			can_work_consec = self.consec_table[n, h]

			already_started = present_hours >= 1
			is_next_hour_present = ((n,h+1,1) in solution 
				or (n,h+1,2) in solution)
			is_previous_hour_present = ((n,h-1,1) in solution 
				or (n,h-1,2) in solution)

			is_next_hour_home = (n,h+1,0) in solution
			is_previous_hour_home = (n,h-1,0) in solution

			break_previous_hour = (n, h-1, 2) in solution
			break_next_hour = (n, h+1, 2) in solution

			# If we already have a assigned state, avoid
			if already_assigned:
				costs[i] = inf
				continue

			if s in {1, 2}: # Present at the hospital

				# If we are already the maximum time at the hospital, avoid
				if present_hours >= p.maxPresence:
					costs[i] = inf
					continue

				if already_started:

					# If already started, and is next and previous hour home,
					# avoid going work again

					# If is adding more consecutive hours, prefer
					if is_next_hour_present or is_previous_hour_present:
						# But only if is some demand at that point
						if demand_left > 0:
							cost -= 333

					# Otherwise, try to avoid
					else:
						costs[i] = inf + 4
						continue

			if s == 0: # Home
				if already_started:

					# Delay going home if is not at the hospital
					if not is_next_hour_present:
						cost += 401

					else:
						cost += 401

				else:
					# Delay if didnt started
					cost += 500

			elif s == 1: # Working

				# Avoid more work than allowed
				if working_time >= p.maxHours:
					costs[i] = inf
					continue

				# Avoid working more than maxConsec hours
				if can_work_consec == 0:
					costs[i] = inf + 2
					continue

				# We want a new nurse proportional to the demand
				if demand_left > 0:
					cost -= demand_left * 100

				# Also try to reach minHours only if already started
				if hours_left > 0 and already_started:
					cost -= hours_left * 51

				# Try to avoid work if there is no demand
				if not already_started and demand_left <= 0:
					cost += 700

			elif s == 2: # Break
				# If was already resting, avoid at all costs
				if break_previous_hour or break_next_hour:
					costs[i] = inf
					continue

				# Only rest if it's better than work
				if already_started:
					cost += 50
				else:
					cost += 700

			costs[i] = cost

		return costs
\end{pycode}


\subsection{Local search}
%
Local search was not used, the greedy construction phase has been optimized to 
find a good solution by using an exhaustive search of possibilities. The 
pseudocode can be found below.
%
\begin{pycode}
def local_search(self, solution):
	while not is_locally_optimal(solution):
		neighbors, costs = find_neigbors(solution)
		index_best = np.argmin(costs)
		solution = neighbors[index_best]
	return solution
\end{pycode}
%

\subsection{Comments}

The implementation of the problem in GRASP can be greatly improved by using a 
representation of the problem similar to the one used in BRGKA.

\section{BRKGA}
%
Genetic algorithms borrow mechanisms from nature, namely the concept of survival 
of the fittest, in order to search a large solution space. A population of 
solutions is evolved over several generations by iteratively breeding new 
solutions by mating or \textit{crossing} individuals from previous generations.  
Biased Randomized-Key Genetic Algorithm (BRKGA) is such an algorithm in which 
certain members of the population are selected for crossover with higher 
probability.

We represent our problem with a set of individuals with an associated 
\textit{chromosome}, each of which encodes a solution.

Each individual or chromosome has associated with it a fitness score, which will 
determine if they are an \textit{elite individual} or \textit{non-elite 
individual}. We can now add a bias to the selection of the parentage for next 
generation individuals the selecting a parent from the elite set with a higher 
probability in order to pass on desirable characteristics.

We used the parameters $p(elite) = 0.1$, $p(mutant) = 0.2$, $p(inherit) = 0.7$ 
and population number=10. The absolute maximum number of iterations is set to 
10000, but if there is no improvement in the best fit, for the last 500 
iterations, an early break happens.

\subsection{Chromosome structure}

The solution is codified by using three scalars $p$, $s$ and $b$ and a vector 
$v$ of size $H$ for each nurse. The scalars codify the number of hours $p$ that 
the nurse is present at the hospital, the first hour $s$ at which is at the 
hospital, and the number of breaks $b$ while is present.

The vector $v$ is used from $1$ to the number of present hours $p$. Each element 
stores a score that will be used to determine the preferred hours in which a 
break should be set. Lower scores will be selected first, filling the breaks up 
to $b$.

The decision whether a nurse works or not, is based on $p$, if the presence 
hours are less than $minHours$ the nurse doesn't start working.

The chromosome structure is then matrix of $N$ rows and $3+H$ columns, but 
flattened into a long vector of size $N \times 3+H$.

\subsection{Decoder}

The decoder part of the algorithm, which is the only part that depends on the 
problem, computes the fitness of each individual in the population. The fitness 
$f(p,s,b,v)$ is a heuristic value that estimates the objective, and has to be 
minimized. Is computed by adding five badness values $BN$, $BD$, $BH$, $BC$ and 
$BF$ together.

In order to compute the values, we first need to determine when each nurse 
works. By using $p$ and $s$ we have the interval that they are present.  And 
using the indices of the first $b$ lower values of $v$, we now where they rest.

The number of nurses needed is the first value $BN$, which takes into account 
that we want to minimize the number of nurses used.

The value $BD$ measures the demand left that is not met. Is computed at each 
hour $h$ as the difference of the problem demand and the nurses offer (the 
working hours provided by the working nurses) as $BD = 100 \sum_h 
\textrm{demand}_h - \textrm{offer}_h$.

The value $BH(n)$ measures the error of each nurse $n$, when they work more than 
$maxHours$ or when they work less than $minHours$, we use $w = p - b$ to compute 
the number of working hours. If they work more hours than allowed, $BH(n)$ is 
set to the number of exceeding hours $w - maxHours$. If they work less than 
allowed, $BH(n)$ is set to the number of hours left $minHours - w$.  Finally, we 
get $BH = \sum_n 200 + 20 BH(n)$.

Note that we use the floating point representation of the values $p$, $s$, $b$ 
and $v$ when they are part of the fitness function. So we can allow that lower 
differences of each gene produce a change in the fitness function. Otherwise, we 
use the integer representation of the parameters.

The value $BC$ test if there is any problem with the maximum number of 
consecutive hours that each nurse is working. For each nurse $n$ the maximum 
number of consecutive hours $M$ is computed, and if is greater than $maxConsec$, 
$BC(n)$ is set to $100(M - maxConsec) - 10b$. We use the breaks to prefer those 
solutions that include more breaks. Finally we have $BC = \sum_n BC(n)$.

The value $BF$ measures the distance of the non-working nurses to the hours that 
have some demand left. The distance is computed for each hour with demand left 
$h$ to the closest hour $c$ that the nurse would start working if $p$ was at 
least $minHours$. Is the start hour $s$ when $s > h$ and the end hour $s+p$ when 
$s+p < h$. So, if the nurse don't cover the hour $h$, $BF(n) = h - c$. However, 
if the nurse could cover the hour, without changing start, we only need to take 
into account the distance to minHours, so $BF(n) = minHours - p$. The final 
value is the sum for all nurses that are non-working, $BF = \sum_n BF(n)$

The final fitness value is just the sum of all the intermediate values:

$$ f(p,s,b,v) = BN + BD + BH + BC + BF $$

The fitness is stored in an array, computed for each individual of the 
population, and returned to the BRKGA algorithm, in order to continue the 
iteration.

\section{Results}

The three solvers were compared solving equal problems, with a big number of 
nurses. The CPLEX solver was limit to 1 minute of execution time, and the best 
solution found was used. In the table~\ref{tab:results} we can see the execution 
time in seconds, as well as the best objective value found, the number of nurses 
needed, for each method. A \texttt{*} symbol is used to represent that the 
method did not found any solution. By setting the seed to the indicated value, a 
reproducible run should produce the same results (time may vary).

\begin{table}[ht]
%\footnotesize
\centering
\begin{tabular}{cccccccc}
\toprule
&&\multicolumn{2}{c}{ILP}
& \multicolumn{2}{c}{GRASP}
& \multicolumn{2}{c}{BRKGA} \\
Seed & Nurses & Obj. & Time & Obj. & Time  & Obj. & Time  \\
\midrule
	8  &     90 &  41  & 61.1 &  52  &  62.9 &  *   &  83.1 \\
	9  &     90 &  39  & 61.2 &   *  & 619.3 &  42  &  71.2 \\
	10 &     90 &  39  & 61.2 &  56  &  62.2 &  42  &  51.6 \\
	11 &     90 &  31  & 61.2 &  43  &  63.4 &  44  &  46.2 \\
	12 &     90 &  39  & 61.2 &  46  & 127.4 &  41  &  96.5 \\
	13 &     90 &  36  & 61.1 &   *  & 638.2 &  *   &  81.3 \\
	14 &     90 &  39  & 61.1 &  51  &  95.3 &  *   &  71.1 \\
	15 &     90 &  31  & 61.2 &  42  &  32.2 &  43  &  65.3 \\
	16 &     90 &  35  & 61.1 &  56  & 189.3 &  44  &  45.5 \\
	17 &     90 &  35  & 61.2 &  57  &  31.5 &  42  &  63.4 \\
	18 &     90 &  37  & 61.8 &   *  & 378.5 &  45  &  54.0 \\
	19 &     90 &  31  & 61.1 &  46  &  31.9 &  33  &  28.3 \\
	20 &     90 &  31  & 61.1 &  43  &  15.0 &  38  &  19.6 \\
	21 &     90 &  33  & 61.1 &   *  & 324.4 &   *  &  54.9 \\
	22 &     90 &  40  & 61.1 &   *  & 479.2 &   *  & 119.4 \\
	23 &     90 &   *  &  1.5 &   *  & 472.3 &   *  &  76.2 \\
	24 &     90 &  32  & 61.1 &  46  &  17.5 &   *  &  26.7 \\
	25 &     90 &  31  & 61.1 &  46  &  17.2 &   *  &  37.0 \\
	26 &     90 &  40  & 61.1 &  60  & 175.3 &  39  &  67.1 \\
	27 &     90 &  35  & 61.1 &   *  & 353.6 &  37  &  48.7 \\
\midrule
	30 &    150 &  60  & 61.9 &      &       &   *  & 236.9 \\
	31 &    150 &  60  & 61.9 &      &       &  68  & 140.0 \\
	32 &    150 &  56  & 61.9 &      &       &  64  & 200.1 \\
	33 &    150 &  66  & 61.9 &      &       &   *  & 186.0 \\
	34 &    150 &  67  & 61.9 &      &       &  72  & 293.6 \\
	35 &    150 &  53  & 61.9 &      &       &  73  & 161.5 \\
	36 &    150 &  75  & 61.9 &      &       &  77  & 152.3 \\
	37 &    150 &  58  & 61.8 &      &       &  63  & 216.3 \\
	38 &    150 &  62  & 61.8 &      &       &  64  & 219.8 \\
\bottomrule
\end{tabular}
\caption{Results with the different methods. Time in seconds.}
\label{tab:results}
\end{table}

We see that the ILP method gets a good solution, even with larger instances. It 
is not able to compute the optimal solution, but the one found has a very low 
objective value. The GRASP method however, not always is able to find a solution 
and usually takes longer. The BRKGA method computes values close to the ILP 
method, in about the same time for 90 nurses. However, is not alway able to find 
one feasible.

\section{Structure of the project}

\subsection{The OPL model}

The ILP problem was first written in OLP by using a small example, and it can be 
found in the \texttt{opl/} folder. The \texttt{proj.mod} file contains the model 
of the problem, and the example data is stored in \texttt{proj.dat}. A small 
python script and a CPLEX batch file help the extraction of the solution. The 
Makefile is designed to use \texttt{oplrun} and \texttt{cplex} to solve the 
problem and present the solution. The \texttt{oplrun} program evaluates the 
model and builds the file \texttt{sol.lp}, which is solved by the \texttt{cplex} 
program. The script \texttt{cplex-save.txt} provide CPLEX the instructions to 
save the solution in \texttt{sol.xml} in order to be read by the python script 
\texttt{cplex-save.txt}:

\begin{textcode}
% cd opl/
% make
...
cplex -f cplex-save.txt

Welcome to IBM(R) ILOG(R) CPLEX(R) Interactive Optimizer 12.6.0.0
...
Root relaxation solution time = 0.00 sec. (0.15 ticks)
...
python read-sol.py
NStart
[[0 0 0 0 1 0]
 [1 0 0 0 0 0]
 [1 0 0 0 0 0]]
NWorking
[[0 0 0 0 1 1]
 [1 1 0 0 0 0]
 [1 1 0 0 0 0]]
NPresent
[[0 0 0 0 1 1]
 [1 1 0 0 0 0]
 [1 1 0 0 0 0]]
\end{textcode}

As we need to run and evaluate a lot of problems, this method is a bit complex, 
as requires the problem to be defined in the OPL data language.  We opted for 
the use of the PuLP python package to communicate with CPLEX, in order to write 
the problem and read the solution as well as the objective value, more easily.

\subsection{The PuLP model}

The PuLP package provides a almost transparent interface to work with multiple 
solvers. We used the CPLEX solver by default, as it seems to be the fastest.

The same constraints in \texttt{opl/proj.mod} were translated into python. The 
file \texttt{src/lp.py} contains the ILP model in python. A class 
\texttt{Solution} provides a generic method \texttt{solve} to be implemented, as 
well as an instance to a problem to be solved.

\subsection{Instance generator}

The instance generator is built in the \texttt{Problem} class, located in 
\texttt{nurses.py}

Random instances of the problem are generated by using the parameters: 
\texttt{seed} the random seed, \texttt{N} the number of nurses and \texttt{H} 
the hours in a day, set by default to 24.

We used a random distribution, with more expected value in the daily hours, 
behaving similarly to a real case. Once a instance problem generated can be 
solved by any solver. An example problem:

\begin{textcode}
% python nurses.py
Problem(seed=1, N=10, H=24)
-------------------------------------------------------
hour  0 0 0 0 0 0 0 0 0 0 1 1 1 1 1 1 1 1 1 1 2 2 2 2 
      0 1 2 3 4 5 6 7 8 9 0 1 2 3 4 5 6 7 8 9 0 1 2 3 
-------------------------------------------------------
dem.  1 1 0 0 1 0 0 2 1 2 2 2 3 2 3 1 1 0 1 0 0 0 1 0 
-------------------------------------------------------
 maxConsec   : 7
 maxPresence : 9
 maxHours    : 8
 minHours    : 8
-------------------------------------------------------
\end{textcode}

%\subsection{Solvers}

\section{How to run}

The ILP, GRASP and BRKGA solvers are coded in a extension of the 
\texttt{Solution} class, placed in \texttt{src/}, in the files \texttt{lp.py}, 
\texttt{grasp.py} and \texttt{brkga.py}, respectively.

Each solver is iteratively chosen to solve the same problem. Once all the 
solutions are computed, a new random problem is generated and solved again. The 
file \texttt{main.py} produces the table~\ref{tab:results}. Can be executed as:
%
\begin{textcode}
% python main.py
\end{textcode}
%
In order to execute the same results, fix the seed to the indicated in the 
table. A total of 10 runs is set by default, automatically incrementing the 
seed, and printing the intermediate table at each row.

\end{document}
